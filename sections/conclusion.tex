\section{Conclusion}\label{sec:conclusion}

We have discussed TensorFlow, a novel open source deep learning library based
on computational graphs. Its ability to perform fast automatic gradient
computation, its inherent support for distributed computation and specialized
hardware as well as its powerful visualization tools make it a very welcome
addition to the field of machine learning. Its low-level programming interface
gives fine-grained control for neural net construction, while abstraction
libraries such as TFLearn allow for rapid prototyping with TensorFlow.  In the
context of other deep learning toolkits such as Theano or Torch, TensorFlow adds
new features and improves on others. Its performance was inferior at first, but
is improving with new releases of the library.

We note that very little investigation has been done in literature to evaluate
TensorFlow's qualities with respect to distributed execution. We esteem this one
of its principle strong points and thus encourage in-depth study by the academic
community in the future.

TensorFlow has gained great popularity and strong support in the open-source
community with many third-party contributions, making Google's move a sensible
decision already. We believe, however, that it will not only benefit its parent
company, but the greater scientific community as a whole; opening new doors to
faster, larger-scale artificial intelligence.

%%% Local Variables:
%%% mode: latex
%%% TeX-master: "../paper"
%%% End:
