\section{Conclusion}\label{sec:conclusion}

We have discussed TensorFlow, a new open source deep learning library based on
computational graphs. Its ability to perform fast automatic gradient
computation, its inherent support for distributed computation and specialized
hardware as well as its powerful visualization tools make it a very welcome
addition to the field of machine learning. We showed how TensorFlow's
programming interface allowed for easy implementation of a simple machine
learning task, forming the basis for more complex models. In the context of
other deep learning toolkits such as Theano or Torch, TensorFlow adds new
features and improves on others. On a qualitative basis, we gave evidence for
the competitive performance of TensorFlow compared to alternative libraries.

TensorFlow has gained great popularity and strong support in the open-source
community with many third-party contributions, making Google's move a sensible
decision already. We believe, however, that it will not only benefit its parent
company, but the greater scientific community as a whole; opening new doors to
faster, larger-scale artificial intelligence.

%%% Local Variables:
%%% mode: latex
%%% TeX-master: "../paper"
%%% End:
