\section{Introduction}

Modern artificial-intelligence systems and machine-learning algorithms have
revolutionized approaches to a myriad of scientific and technological challenges
in a variety of fields. We can observe an immense proliferation in the quality
of state-of-the-art computer-vision and image-recognition, natural language
processing, speech recognition and other techniques. However, the major and most
obvious benefactor of this revolution is mankind itself. Personalized digital
assistants, recommendations on e-commerce platforms, financial fraud detection,
customized web-search results and social-network feeds as well as novel
breakthroughs in genomics have all been improved, if not enabled, by current
machine learning methods.

A particular branch of machine-learning, \emph{deep-learning}, has proven
especially effective in recent years. Deep-learning may be defined as a family
of representation-learning algorithms employing complex neural-network
architectures with a high number of hidden layers, each composed of simple but
non-linear transformations to the input data. Given enough such transformation
modules, very complex functions may be modeled to solve classification,
regression, transcription and numerous other learning tasks \cite{nature2015}.

It is noteworthy that the rise in popularity of deep-learning can be traced back
to only the last few years, enabled primarily by the discovery of new algorithms
such as the \emph{rectified linear unit} (ReLU) \cite{relu} activation function
or \emph{dropout} as a regularization technique \cite{dropout}; the greater
availability of large data-sets, containing more training examples and lastly
the efficient use of graphical processing units (GPUs) and massively parallel
commodity hardware to train deep-learning models on these equally massive
data-sets \cite{nature2015, rampasek}.

While deep-learning algorithms and individual architectural components such as
representation transformations, activation functions or regularization methods
may initially be expressed in mathematical notation, they must eventually be
transcribed into a computer program for real-world usage. For this purpose,
there exist a number of open-source as well as commercial machine-learning
software libraries and frameworks. Among these are Theano \cite{theano}, Torch
\cite{torch}, scikit-learn \cite{scikit} and many more, which we review in
further detail in Section II of this paper. In November 2015, this list was
extended by \emph{TensorFlow}, a novel machine-learning software library
released by Google \cite{tensorflow}. As per the initial publication, TensorFlow
aims to be ``an interface for expressing machine learning algorithms'' in
``large-scale [\dots] on heterogeneous distributed systems'' \cite{tensorflow}.

The remainder of this paper aims to give a thorough review of TensorFlow and put
it in context of the current state of machine-learning. In detail, the paper is
further structured as follow. Section I will provide a brief overview and
history of machine-learning software libraries, listing but not comparing
projects similar to TensorFlow. Subsequently, Section II will discuss in depth
the computational paradigms underlying TensorFlow. Section III will then move to
explaining the current implementation's programming interface in the various
programming languages supported. In the following, Section IV provides a
qualitative as well as quantitative comparison of TensorFlow and other
\emph{deep-learning} libraries. Before concluding our review in Section VI, we
also examine current real-world use-cases of and experiences with TensorFlow in
Section V.

%%% Local Variables:
%%% mode: latex
%%% TeX-master: "paper"
%%% End:
