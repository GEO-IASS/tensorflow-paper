\section{History of Machine Learning Libraries}

In this section, we aim to give a brief overview and key milestones in the
history of machine-learning software libraries. We begin with a review of
libraries suitable for a wide range of machine-learning and data-analysis
purposes, reaching back more than 20 years. We then provide a more focused
review of recent programming frameworks suited especially to the task of deep
learning. We wish to emphasize that this section does in no way compare or even
mention TensorFlow, as we have dedicated Section IV for this specific purpose.

\subsection{General Machine Learning}

In the following paragraphs we aim to list and briefly review a small set of
\emph{general machine-learning libraries} in chronological order. With
\emph{general}, we mean to describe any particular library whose common
use-cases in the machine-learning and data-science community include \emph{but
  are not limited to} deep-learning. As such, these libraries may be used for
statistical analysis, clustering, dimensionality reduction, structured
prediction, anomaly detection, shallow (as opposed to deep) neural networks and
other interests.

We begin our review with a library released 21 years before TensorFlow:
\emph{MLC++} \cite{mlcpp}. MLC++ is a software library developed in the C++
programming language providing algorithms alongside a comparison-framework for a
number of of data-mining, statistical analysis as well as pattern-recognition
techniques. It was originally developed at Stanford University in 1994 and is
now owned and maintained by Silicon Graphics, Inc
(SGI\footnote{https://www.sgi.com/tech/mlc/}). To the best of our knowledge,
MLC++ is the oldest machine-learning library available today.

Following MLC++ in the chronological order,
\emph{OpenCV}\footnote{http://opencv.org} (\textbf{Open}
\textbf{C}omputer-\textbf{V}ision) was released in the year 2000 by Bradski et
al. \cite{opencv}. It is aimed primarily at solving learning-tasks in the field
of computer-vision and image-recognition, including a collection of algorithms
for face-recognition, object-identifcation, 3D-model extraction and other
purposes. It released under a BSD-license and provides interfaces in multiple
programming languages such as C++, Python, MATLAB, among others.

Another machine-learning library we wish to mention is
\emph{scikit-learn}\footnote{http://scikit-learn.org/stable/} \cite{scikit}. The
scikit-learn project was originally developed by David Cournapeu as part of the
Google Summer of Code program\footnote{https://summerofcode.withgoogle.com} in
2008. It is an open-source machine-learning library written in Python, on top of
the NumPy, SciPy and matplotlib Python-libraries and useful for a large class of
both supervised and unsupervised learning problems.

The \emph{Accord.NET}\footnote{http://accord-framework.net/index.html} library
stands apart from the aforementioned examples in that it is written in the C\#
(``C-Sharp'') programming language. Released in 2008, it is composed not only of
a variety of machine-learning algorithms, but also signal-processing modules for
speech and image recognition \cite{accord}.

\emph{Massive Online Analysis}\footnote{http://moa.cms.waikato.ac.nz} (MOA) is
an open-source framework for online and offline analysis of massive, potentially
infinite, data-\emph{streams}. MOA includes a variety of tools for
classification, regression, recommender systems and further domains. It is
written in the Java programming language and is maintained by staff of the
University of Waikato, New Zealand. It was conveived in 2010 \cite{moa}.

The \emph{Mahout}\footnote{http://mahout.apache.org} project, part of Apache
Software Foundation\footnote{http://www.apache.org}, is a Java programming
environment for scalable machine-learning applications, built on top of the
Apache Hadoop\footnote{http://hadoop.apache.org} platform. It allows for
analysis of large datasets distributed in the Hadoop Distributed File System
(HDFS) using the \emph{MapReduce} programming paradigm and includes
machine-learning algorithms for classification, clustering and filtering.

\emph{Pattern}\footnote{http://www.clips.ua.ac.be/pages/pattern} is a Python
machine-learning module we include in our list due to its rich set of
\emph{web-mining} facilities. It includes not only general machine learning
algorithms (e.g. clustering, classification or nearest-neighbour search) and
natural language processsing methods (e.g. n-gram search or sentiment analysis),
but also a web-crawler that can, for example, fetch Tweets and Wikipedia
entries, facilitating quick data-analysis on these sources. It was published by
the University of Antwerp in 2012 and is open-source.

Lastly, \emph{Spark MLlib}\footnote{http://spark.apache.org/mllib} is an
open-source machine-learning and data-analysis platform released in 2015 and
built on top of the Apache Spark\footnote{http://spark.apache.org/} project
\cite{spark}, a fast cluster-computing system. Similar to Apache Mahout, it aims
to support processing of large-scale \emph{distributed} datasets and training of
machine-learning models across a cluster of commodity hardware. For this, it
includes classification, regression, clustering and other machine-learniing
algorithms \cite{mllib}.

\subsection{Deep Learning}

While the software libraries mentioned in the previous section are useful for a
great variety of different machine-learning and statistical analysis tasks, the
following paragraphs list software frameworks especially effective in training
deep-learning models.

The first and oldest framework in our list suited to the development and
training of deep neural networks is \emph{Torch}\footnote{http://torch.ch},
released already in 2002 \cite{torch}. Torch consisted originally of a pure C++
implementation and interface. Today, its core is implemented in C/CUDA while it
exposes an interface in the Lua\footnote{https://www.lua.org} scripting
language. For this, Torch makes use of a LuaJIT (just-in-time) compiler to
connect Lua routines to the underlying C implementations. It includes, inter
alia, numerical optimization routines, neural network models as well as
general-purpose n-dimensional array (tensor) objects.

\emph{Theano}\footnote{http://deeplearning.net/software/theano/}, released in
2008 \cite{theano}, is another noteworthy deep-learning library. We note that
while Theano enjoys greatest popularity among the machine-learning community, it
is, in essence, not a machine-learning library at all. Rather, it is a
programming framework that allows users to declare mathematical expressions
\emph{symbolically}, as computational graphs wich may be optimized, eventually
compiled and finally executed in a fast and efficient manner on either CPU or
GPU devices. As such, \cite{theano} labels Theano a ``mathematical compiler''.

\emph{Caffe}\footnote{http://caffe.berkeleyvision.org} is an open-source
deep-learning library maintained by the Berkely Vision and Learning Center
(BVLC), released in 2014 under a BSD-License \cite{caffe}. It is implemented in
C++ and uses neural-network layers as its basic computational building blocks
(as opposed to Theano and others, where the user must define individual
mathematical operations making up layers). A deep-learning model, consisting of
many such layers, is stored in the Google Protocol Buffer data-representation
format. While models may manually be defined in this Protocol Buffer
``language'', there exist bindings to Python and MATLAB to generate them
programmatically. Caffe is especially well suited to the development and
training of \emph{convolutional neural networks} (CNNs or ConvNets), used
extensively in the domain of image-recognition.

While the aforementioned machine-learning frameworks allowed for the definition
of deep-learning models in Python, MATLAB and Lua, the
\emph{Deeplearning4J}\footnote{http://deeplearning4j.org} (DL4J) library enables
also the Java programmer to create deep neural-networks. DL4J includes
functionality to create Restricted Boltzmann machines, convolutional and
recurrent neural networks, deep belief networks and other types of deep-learning
models. Moreover, DL4J enables horizontal scalabality using distributed
computing platforms such as Apache Hadoop or Spark. It was released in 2014 by
Adam Gibson, under an Apache 2.0 open-source license.

Lastly, we add the NVIDIA Deep Learning
SDK\footnote{https://developer.nvidia.com/deep-learning-software} to to this
list. Its main goal is to maximize the perfomance of deep-learning algorithms on
(NVIDIA) graphical-processing-units (GPUs). The SDK consists of three core
modules. The first, \emph{cuDNN}, provides high-performance GPU implementations
for deep-learning algorithms such as convolutions, activations functions and
tensor transformations. The second is a linear-algebra library, \emph{cuBLAS},
enabling GPU-accelerated mathematical operations on n-dimensional
arrays. Lastly, \emph{cuSPARSE} includes a set of routines for \emph{sparse}
matrices tuned for high efficiency on GPUs. While programming may be done in
these libraries directly, there exist also bindings to other deep-learning
libraries, such as Torch\footnote{https://github.com/soumith/cudnn.torch}.

%%% Local Variables:
%%% mode: latex
%%% TeX-master: "../paper"
%%% End: