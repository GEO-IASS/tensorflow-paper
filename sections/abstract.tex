\begin{abstract}

  Deep learning is a branch of artificial intelligence employing deep neural
  network architectures that has significantly advanced the state-of-the-art in
  computer vision, speech recognition, natural language processing and other
  domains. In November 2015, Google released \emph{TensorFlow}, an open source
  deep learning software library for defining, training and deploying machine
  learning models. In this paper, we review TensorFlow and put it in context of
  modern deep learning concepts and software.\vspace{0.5cm}

  We discuss its basic computational
  paradigms and distributed execution model, its programming interface as well
  as accompanying visualization toolkits. We then compare TensorFlow to
  alternative libraries such as Theano, Torch or Caffe on a qualitative as well
  as quantitative basis and finally comment on observed use-cases of TensorFlow
  in academia and industry.\vspace{0.5cm}

  We discuss why TensorFlow's basic computational paradigms, distributed
  execution model and strong hardware support are well suited to deep
  learning. Next to its flexible programming interface, we comment on
  TensorFlow's powerful visualization tools that allow easy inspection and
  debugging of complex neural network architectures. We show that while
  TensorFlow's performance initially compared worse to alternative deep learning
  frameworks such as Torch, Theano or Caffe, the trend is in favor of
  TensorFlow. Given its unique many-machine capabilities, which we examine in
  detail in this paper, we estimate that TensorFlow's quantitative ranking will
  improve further once these features are harnessed. Since its release,
  TensorFlow has already found adoption in academia and industry, for which we
  give a handful of examples. In total, this paper gives convincing support as
  to why TensorFlow has all the potential to improve existing machine learning
  systems, open new doors for scientific research and enable faster,
  larger-scale artificial intelligence in the future.

\end{abstract}

%%% Local Variables:
%%% mode: latex
%%% TeX-master: "../paper"
%%% End:
