\begin{abstract}

  Deep learning is a branch of artificial intelligence employing deep neural
  network architectures that has significantly advanced the state-of-the-art in
  computer vision, speech recognition, natural language processing and other
  domains. In November 2015, Google released \emph{TensorFlow}, an open source
  deep learning software library for defining, training and deploying machine
  learning models. In this paper, we review TensorFlow and put it in context of
  modern deep learning concepts and software. We discuss its basic computational
  paradigms and distributed execution model, its programming interface as well
  as accompanying visualization toolkits. We then compare TensorFlow to
  alternative libraries such as Theano, Torch or Caffe on a quantitative basis
  and finally comment on observed use-cases of TensorFlow in academia and
  industry.\vspace{0.5cm}
\end{abstract}

%%% Local Variables:
%%% mode: latex
%%% TeX-master: "../paper"
%%% End:
