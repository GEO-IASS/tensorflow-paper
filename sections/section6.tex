\section{Use-Cases of TensorFlow Today}\label{sec:uses}

In this section, we investigate where TensorFlow is already in use today. Given
that TensorFlow was released only little over 6 months ago as of this writing,
its adoption in academia and industry is not yet widespread. Migration from an
existing system based on some other library within small and large organizations
necessarily takes time and consideration, so this is not unexpected. The one
exception is, of course, Google, which has already deployed TensorFlow for a
variety of learning tasks \cite{drugs, phones, emails, deepmind, inception}. We
begin with a review of selected mentions of TensorFlow in literature. Then, we
discuss where and how TensorFlow is used in industry.

\subsection{In Literature}\label{sec:uses-lit}

The first noteworthy mention of TensorFlow is \cite{szegedy2016}, published by
Szegedy, Ioffe and Vanhoucke of the Google Brain Team in February 2016. In their
work the authors use TensorFlow to improve on the \emph{Inception} model
\cite{inception}, which achieved best performance at the 2014 ImageNet
classification challenge. The authors report a 3.08\% top-5 error on the
ImageNet test set.

In \cite{drugs}, Ramsundar et al. discuss massively ``multitask networks for
drug discovery'' in a joint collaboration work between Stanford University and
Google, published in early 2016. In this paper, the authors employ deep neural
networks developed with TensorFlow to perform virtual screening of potential
drug candidates. This is inteded to aid pharmaceutical companies and the
scientific community in finding novel medication and treatments for human
diseases.

August and Ni apply TensorFlow to create recurrent neural networks to optimize
dynamic decoupling, a technique for suppressing errors in quantum memory
\cite{august}. With this, the authors aim to preserve the coherence of quantum
states, which is one of the primary requirements for building universal quantum
computers.

Lastly, \cite{barzdins2016} investigates the use of sequence to sequence neural
translation models for natural language processing of multilingual media
sources. For this, Barzdins et al. use TensorFlow with a sliding-window approach
to character-level English to Latvian translation of audio and video
content. The authors use this to segment TV and radio programs and cluster
individual stories.

\subsection{In Industry}\label{sec:uses-industry}

Adoption of TensorFlow in industry is currently limited only to Google, at least
to the extent that is publicly known. We have found no evidence of any other
small or large corporation stating its use of TensorFlow. As mentioned, we link
this to TensorFlow's late release. Moreover, it is obvious that many companies
would not make their machine learning methods public even if they do use
TensorFlow. For this reason, we will review uses of TensorFlow only within
Google, Inc.

Recently, Google has begun augmenting its core search service and accompanying
\emph{PageRank} algorithm\cite{pagerank} with a system called
\emph{RankBrain}\cite{rankbrain}, which makes use of TensorFlow. RankBrain uses
large-scale distributed deep neural networks for search result
ranking. According to \cite{rankbrain}, more than 15 percent of all search
queries received on www.google.com are new to Google's system. RankBrain can
suggest words or phrases with similar meaning for unknown parts of such queries.

Another area where Google applies deep learning with TensorFlow is smart email
replies \cite{emails}. Google has investigated and already deployed a feature
whereby its email service \emph{Inbox} suggests possible replies to received
email. The system uses recurrent neural networks and in particular LSTM modules
for sequence-to-sequence learning and natural language understanding. An encoder
maps a corpus of text to a ``thought vector'' while a decoder synthesizes
syntactically and semantically correct replies from it, of which a selection is
presented to the user.

In \cite{phones} it is reported how Google employs convolutional neural networks
for image recognition and automatic text translation. As a feature integrated
into its Google Translate mobile app, text in a language foreign to the user is
first recognized, then translated and finally rendered on top of the original
image. In this way, for example, street signs can be translated. \cite{phones}
notes especially the challenge of deploying such a system onto low-end phones
with slow network connections. For this, small neural networks were used and
trained to discover only the most essential information in order to optimize
available computational resources.

Lastly, we make note of the decision of Google DeepMind, an AI division within
Google, to move from Torch7 to TensorFlow \cite{deepmind}. A related source,
\cite{tpu}, states that DeepMind made use of TensorFlow for its
\emph{AlphaGo}\footnote{https://deepmind.com/alpha-go} model, alongside Google's
newly developed Tensor Processing Unit (TPU), which was built to integrate
especially well with TensorFlow. In a correspondence of the authors of this
paper with a member of the Google DeepMind team, the following four reasons were
revealed to us as to why TensorFlow is advantageous to DeepMind:

\begin{enumerate}
\item TensorFlow is included in the Google Cloud
  Platform\footnote{https://cloud.google.com/compute/}, which enables easy
  replication of DeepMind's research.
\item TensorFlow's support for TPUs.
\item TensorFlow's main interface, Python, is one of the core languages at
  Google, which implies a much greater internal toolset than for Lua.
\item The ability to run TensorFlow on many GPUs.
\end{enumerate}

%%% Local Variables:
%%% mode: latex
%%% TeX-master: "../paper"
%%% End: